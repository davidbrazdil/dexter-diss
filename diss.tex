% The master copy of this demo dissertation is held on my filespace
% on the cl file serve (/homes/mr/teaching/demodissert/)

% Last updated by MR on 2 August 2001

\documentclass[12pt,twoside,notitlepage]{report}

\usepackage{a4}
\usepackage{verbatim}
\usepackage{units}
\usepackage{hyperref}
\usepackage{array}

\input{epsf}                            % to allow postscript inclusions
% On thor and CUS read top of file:
%     /opt/TeX/lib/texmf/tex/dvips/epsf.sty
% On CL machines read:
%     /usr/lib/tex/macros/dvips/epsf.tex



\raggedbottom                           % try to avoid widows and orphans
\sloppy
\clubpenalty1000%
\widowpenalty1000%

\addtolength{\oddsidemargin}{6mm}       % adjust margins
\addtolength{\evensidemargin}{-8mm}

\renewcommand{\baselinestretch}{1.1}    % adjust line spacing to make
                                        % more readable
                                        
\newcommand{\strAuthor}{David Brazdil}
\newcommand{\strCollege}{Trinity Hall}
\newcommand{\strTitle}{Taint-based Data Flow Analysis on Android}
\newcommand{\strExamination}{Computer Science Tripos, Part II}
\newcommand{\strYear}{June 2013}
\newcommand{\strSupervisor}{Dr A. Beresford}

\title{\strTitle}
\author{\strAuthor}

\begin{document}

\bibliographystyle{plain}


%%%%%%%%%%%%%%%%%%%%%%%%%%%%%%%%%%%%%%%%%%%%%%%%%%%%%%%%%%%%%%%%%%%%%%%%
% Title


\pagestyle{empty}

\hfill{\LARGE \bf \strAuthor}

\vspace*{60mm}
\begin{center}
\Huge
{\bf \strTitle} \\
\vspace*{5mm}
Diploma in Computer Science \\
\vspace*{5mm}
Trinity Hall \\
\vspace*{5mm}
\today  % today's date
\end{center}

\cleardoublepage

%%%%%%%%%%%%%%%%%%%%%%%%%%%%%%%%%%%%%%%%%%%%%%%%%%%%%%%%%%%%%%%%%%%%%%%%%%%%%%
% Proforma, table of contents and list of figures

\setcounter{page}{1}
\pagenumbering{roman}
\pagestyle{plain}

\chapter*{Proforma}

{\large
\begin{tabular}{ll}
Name:               & \bf \strAuthor                       \\
College:            & \bf \strCollege                     \\
Project Title:      & \bf \strTitle \\
Examination:        & \bf \strExamination, \strYear        \\
Word Count:         & \bf 1587\footnotemark[1] \\
Project Originator: & \strSupervisor                    \\
Supervisor:         & \strSupervisor                    \\ 
\end{tabular}
}
\footnotetext[1]{This word count was computed
by {\tt detex -n diss.tex | tr -cd '0-9A-Za-z $\tt\backslash$n' | wc -w}
}
\stepcounter{footnote}


\section*{Original Aims of the Project}

To write a demonstration dissertation\footnote{A normal footnote without the
complication of being in a table.} using \LaTeX\ to save
student's time when writing their own dissertations. The dissertation
should illustrate how to use the more common \LaTeX\ constructs. It
should include pictures and diagrams to show how these can be
incorporated into the dissertation.  It should contain the entire
\LaTeX\ source of the dissertation and the Makefile.  It should
explain how to construct an MSDOS disk of the dissertation in
Postscript format that can be used by the book shop for printing, and,
finally, it should have the prescribed layout and format of a diploma
dissertation.


\section*{Work Completed}

All that has been completed appears in this dissertation.

\section*{Special Difficulties}

Learning how to incorporate encapulated postscript into a \LaTeX\
document on both CUS and Thor.
 
\newpage
\section*{Declaration}

I, [Name] of [College], being a candidate for Part II of the Computer
Science Tripos [or the Diploma in Computer Science], hereby declare
that this dissertation and the work described in it are my own work,
unaided except as may be specified below, and that the dissertation
does not contain material that has already been used to any substantial
extent for a comparable purpose.

\bigskip
\leftline{Signed [signature]}

\medskip
\leftline{Date [date]}

\cleardoublepage

\tableofcontents

\listoffigures

\newpage
\section*{Acknowledgements}

This document owes much to an earlier version written by Simon Moore
\cite{moore95}.  His help, encouragement and advice was greatly 
appreciated.

%%%%%%%%%%%%%%%%%%%%%%%%%%%%%%%%%%%%%%%%%%%%%%%%%%%%%%%%%%%%%%%%%%%%%%%
% now for the chapters

\cleardoublepage        % just to make sure before the page numbering
                        % is changed

\setcounter{page}{1}
\pagenumbering{arabic}
\pagestyle{headings}

\chapter{Introduction}


\cleardoublepage
\chapter{Preparation}

\section{Dalvik Virtual Machine}

Dalvik is an open-source virtual machine and an essential component of the Android operating system. It was developed specifically for use on mobile devices such as smartphones and tablets, and its design was optimised for running on battery-powered systems with low-performance processor and limited memory. 

Built around the Apache Harmony project, Dalvik runs programs that are compatible with a subset of the Java runtime framework. Applications for Dalvik are therefore typically written in Java, compiled to a set of JVM bytecode \textit{.class} files, and then converted to a single Dalvik-compatible \textit{.dex} file using tools in the Android SDK. Thanks to Dalvik's JIT compiler, applications for Android run almost natively while staying independent of the underlying architecture.

\subsection{Application Package Files}

Applications for Android are distributed in Application Package files (APK). These are ZIP files signed by the publisher, which contain:
\begin{itemize}
\item the executable \textbf{classes.dex} file
\item application resources (images, UI layouts, ...)
\item native code binaries
\item manifest
\end{itemize}

The compulsory manifest file contains essential information about the application to the operating system. Among other things, this includes its unique package name, list of permissions requested by the application, and a list of all its entry points.

\subsection{Bytecode}

\subsubsection{Registers}

Unlike the stack-based JVM, Dalvik uses a register-based programming model with 32-bit register width and variable-length instructions. 64-bit values are stored in two adjecent registers.

Each method can use up to 65,536 virtual registers. Actual mapping to hardware registers depends on the machine architecture. On a typical Android device, the first 16 virtual registers would map to the 16 registers available in the ARM instruction set, and rest would be stored in memory. Depending on the number of arguments and their length, instructions can address either the first 16, first 256, or all 65k virtual registers. If an instruction needs to address a register out of the available range, the register contents are expected to get moved to a lower register first.

\subsubsection{Syntax of Assembly}
Assembly syntax follows the \textit{dest-then-source} ordering of arguments. Instructions can also contain immediate values, like numeric constants or pointers to objects in constant pools. For example, instruction
\begin{center}
\verb$add-int/lit16 v2, v8, #1234$
\end{center}
adds 1234 to the value stored in virtual registers 8, and stores the result into register 2. 

Complete list of supported instructions is available on the Android Developer website.

The example code in this dissertation will loosely follow this syntax. However, since Dexter operates on a representation with unlimited number of registers which get reallocated to the 16-bit range at a later stage, written rules of instrumentation will treat registers like variables, e.g.
\begin{center}
\verb$aget-object rResult, rArray, rIndex$
\end{center}
retrieves element of a given array at a given index.

For better clarity, the first letters of register names will represent their origin. Names of registers used in the original code will start with \verb$r$, registers added for tainting with \verb$t$, and other with \verb$p$.

In Dalvik, only the lower of two registers representing a 64-bit value is addressed by the instruction. Higher registers are addressed implicitly. For example,
\begin{center}
\verb$move-wide v2, v3$
\end{center}
moves a wide value from registers \verb$v3$ and \verb$v4$ to registers \verb$v2$ and \verb$v3$. Because Dexter does not reference registers by their number, all actual arguments of instructions will be written out and registers forming a wide argument will be separated by a bar, e.g.
\begin{center}
\verb$move-wide rTo1|rTo2, rFrom1|rFrom2$
\end{center}

\subsubsection{Instruction Constraints}

\section{Application Installation on Android}

\section{Taint-based Flow Analysis}

\section{Requirements Analysis}
look at AdCache

\section{Software Engineering}

\subsection{Approach}

\subsection{Android Genome Project}

\section{Tools}

\subsection{Programming Languages}

\subsection{Software Libraries}

\subsection{Android SDK}

\subsection{Version Control}

\subsection{Evaluation Tools}

\cleardoublepage
\chapter{Implementation}




\cleardoublepage
\chapter{Evaluation}


\cleardoublepage
\chapter{Conclusion}



\cleardoublepage

%%%%%%%%%%%%%%%%%%%%%%%%%%%%%%%%%%%%%%%%%%%%%%%%%%%%%%%%%%%%%%%%%%%%%
% the bibliography

\addcontentsline{toc}{chapter}{Bibliography}
\bibliography{refs}
\cleardoublepage

%%%%%%%%%%%%%%%%%%%%%%%%%%%%%%%%%%%%%%%%%%%%%%%%%%%%%%%%%%%%%%%%%%%%%
% the appendices
\appendix

\chapter{Latex source}

\section{diss.tex}
{\scriptsize\verbatiminput{diss.tex}}

\section{proposal.tex}
{\scriptsize\verbatiminput{proposal.tex}}

\section{propbody.tex}
{\scriptsize\verbatiminput{propbody.tex}}



\cleardoublepage

\chapter{Makefile}

\section{\label{makefile}Makefile}
{\scriptsize\verbatiminput{makefile.txt}}

\section{refs.bib}
{\scriptsize\verbatiminput{refs.bib}}


\cleardoublepage

\chapter{Bytecode for the Dalvik VM}

% \input{dalvik-bytecode}


\cleardoublepage

\chapter{Project Proposal}


% Draft #1 (final?)

\vfil

\centerline{\Large Computer Science Project Proposal}
\vspace{0.4in}
\centerline{\Large How to write a dissertation in \LaTeX\ }
\vspace{0.4in}
\centerline{\large M. Richards, St John's College}
\vspace{0.3in}
\centerline{\large Originator: Dr M. Richards}
\vspace{0.3in}
\centerline{\large 14$^{th}$ October 2011}

\vfil


\noindent
{\bf Project Supervisor:} Dr M. Richards
\vspace{0.2in}

\noindent
{\bf Director of Studies:} Dr M. Richards
\vspace{0.2in}
\noindent
 
\noindent
{\bf Project Overseers:} Dr~F.~H.~King  \& Dr~A.~W.~Moore


% Main document

\section*{Introduction, The Problem To Be Addressed}


Many students write their CST dissertations in \LaTeX\ and
spend a fair amount of time learning just how to do that. The purpose of 
this project is to write a demonstration dissertation that explains in
detail how it done.  

This core proposal document will be augmented by a separately-printed
cover sheet at the front and a resource form at the end.  Additional
sheets for risk assessment and human resources may also need to be included.

This document will repeat much of the material that is summarised on the additional sheets.

\section*{Starting Point}

{\em Describe existing state of the art, previous work in this area, libraries and databases to be used.
Describe the state of any existing codebase that is to be built on.  }

I am already able to write prose using the English language. I have an online dictionary. etc..

\section*{Resources Required}

{\em A note of the resources required and confirmation of access.}

For this project I shall mainly use my own quad-core computer that runs Fedora Linux. Backup
will be to github and/or to an SVN repository on an external hard disk that is dumped to writable CD/DVD media.
I have another similar computer to hand should my main machine suddenly fail.
I require no other special resources.

\section*{Work to be done}

{\em Describe the technical work.}

The project breaks down into the following sub-projects:

\begin{enumerate}

\item The construction of a skeleton dissertation with the required 
structure. This involves writing the Makefile and makeing dummy files
for the title page, the proforma, chapters 1 to 5, the appendices and
the proposal.

\item Filling in the details required in the cover page and proforma.

\item Writing the contents of chapters 1 to 5, including examples
of common \LaTeX\ constructs.

\item Adding a example of how to use floating figures and encapsulated
postscript diagrams.

\end{enumerate}

\section*{Success Criterion for the Main Result}


The project will be a success if I have a completed dissertation with the correct chapter
titles and I have achieved my other success criterion, which is to blah ...



\section*{Possible Extensions}

{\em Potential further envisaged evaluation metrics or extensions.}

If I achieve my main result early I shall try the following alternative experiment or method of evaluation ...


\section*{Timetable: Workplan and Milestones to be achieved.}


{\em Perhaps list ten or so  two-week work-packages.}

Planned starting date is 16/10/2011.

\begin{enumerate}

\item {\bf Michaelmas weeks 2-4} Learn to use X. Read book Y. Read papers Z.

\item {\bf Michaelmas weeks 5-6} Do preliminary test of Q.

\item {\bf Michaelmas weeks 7-8} Start implementation of main task A.

\item {\bf Michaelmas vacation} Finish A and start main task B.

\item {\bf Lent weeks 0-2} Write progress report. Generate corpus of test examples. Finish task B.  

\item {\bf Lent weeks 3-5} Run main experiments and achieve working project.

\item {\bf Lent weeks 6-8} Second main deliverable here.

\item {\bf Easter vacation:} Extensions and writing dissertation main chapters.

\item {\bf Easter term 0-2:}  Further evaluation and complete dissertation.

\item {\bf Easter term 3:} Proof reading and then an early submission so as to concentrate on examination revision.

\end{enumerate}


 



\end{document}
